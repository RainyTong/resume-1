%-------------------------
% Resume in Latex
% Author : Yutong Wang
% License : MIT
%------------------------
\documentclass[letterpaper,11pt]{article}
\usepackage{latexsym}
\usepackage[empty]{fullpage}
\usepackage{titlesec}
\usepackage{marvosym}
\usepackage[usenames,dvipsnames]{color}
\usepackage{verbatim}
\usepackage{enumitem}
\usepackage[hidelinks]{hyperref}
\usepackage{fancyhdr}
\usepackage[english]{babel}
\pagestyle{fancy}
\fancyhf{} % clear all header and footer fields
\fancyfoot{}
\renewcommand{\headrulewidth}{0pt}
\renewcommand{\footrulewidth}{0pt}
% Adjust margins
\addtolength{\oddsidemargin}{-0.5in}
\addtolength{\evensidemargin}{-0.5in}
\addtolength{\textwidth}{1in}
\addtolength{\topmargin}{-.5in}
\addtolength{\textheight}{1.0in}
\urlstyle{same}
\raggedbottom
\raggedright
\setlength{\tabcolsep}{0in}
% Sections formatting
\titleformat{\section}{
  \vspace{-4pt}\scshape\raggedright\large
}{}{0em}{}[\color{black}\titlerule \vspace{-5pt}]
% Hyperlink color
\usepackage{hyperref}
\hypersetup{
    colorlinks=true,
    linkcolor=blue,
    filecolor=blue,      
    urlcolor=blue,
    citecolor=blue,
}
%-------------------------
% Custom commands
\newcommand{\resumeItem}[2]{
  \item\small{
    \textbf{#1}{: #2 \vspace{-2pt}}
  }
}
\newcommand{\resumeSubheading}[6]{
  \vspace{-1pt}\item
    \begin{tabular*}{0.97\textwidth}[t]{l@{\extracolsep{\fill}}r}
      \textbf{#1} & #2 \\
%       \textbf{\small#3} & \textbf{\small #4} \\
      #3 & #4 \\
      \textbf{#5} & \textbf{#6} \\
    \end{tabular*}\vspace{-5pt}
}

\newcommand{\resumeSubItem}[2]{\resumeItem{#1}{#2}\vspace{-4pt}}
\renewcommand{\labelitemii}{$\circ$}
\newcommand{\resumeSubHeadingListStart}{\begin{itemize}[leftmargin=*]}
\newcommand{\resumeSubHeadingListEnd}{\end{itemize}}
\newcommand{\resumeItemListStart}{\begin{itemize}}
\newcommand{\resumeItemListEnd}{\end{itemize}\vspace{-5pt}}
%-------------------------------------------
%%%%%%  CV STARTS HERE  %%%%%%%%%%%%%%%%%%%%%%%%%%%%
\begin{document}
%----------HEADING-----------------
\begin{tabular*}{\textwidth}{l@{\extracolsep{\fill}}r}
  \textbf{\Large Yutong Wang} & Email : yutongwang1998@gmail.com\\
  \href{https://rainytong.github.io/}{https://rainytong.github.io/} & Mobile : +86 15610502699 \\
\end{tabular*}
%-----------EDUCATION-----------------
\section{Education}
  \resumeSubHeadingListStart
    \resumeSubheading
      {Southern University of Science and Technology(\href{https://www.sustech.edu.cn/en/}{SUSTech})}{Shenzhen, China}
      {B.S. degree in Computer Science}{Sept. 2016 -- Jun. 2020(Expected)}
      {}{}
      \begin{itemize}
      \item GPA(total): 89/100, 3.72/4.0
      \\ 
      GPA(sophomore\&junior): 92/100, 3.84/4.0  
  
      \item Annual Outstanding Student Scholarship: 2017-2018, 2018-2019
      
      \item Selected Math\&CS courses with grade A/A+: Linear Algebra, Probability and Statistics, Computer Organization, Computer Networks, Software Engineering
       \end{itemize}
       
    \resumeSubheading
      {University of California, Irvine}{Irvine, California}
      {UCInspire Undergraduate Research Program}{Jul. 2019 -- Sept. 2019}
      {}{}
       \begin{itemize}
        \item GPA(research evaluation): 4.0/4.0
       \end{itemize}
  \resumeSubHeadingListEnd

%-----------RESEARCH INTERESTS---------
\section{Research Interests}
Data Mining, Adversarial Machine Learning, Cybersecurity


%-----------EXPERIENCE-----------------
\section{Research Experience}
  \resumeSubHeadingListStart
    \resumeSubheading
      {\href{https://mine.kaust.edu.sa/Pages/Home.aspx}{MINE Lab}, King Abdullah University of Science and Technology}{Thuwal, Saudi Arabia}
      {Research Intern, Advisor: \href{https://mine.kaust.edu.sa/Pages/ZhangX.aspx}{Professor Xiangliang Zhang}}{Sept. 2019 - Jan. 2020}
      {\textit{Discrete Adversarial Attacks on Medical Predictive Models}}{}
      
        \begin{itemize}
          \item Achieved a higher performance on the state-of-the-art LSTM medical predictive model via designing embeddings for discrete medical knowledge combined with attention mechanism
          
          \item Leveraged heuristic clustering algorithms to analyze the quality of embedding learned for prior medical knowledge
       
          \item Analyzed issues and challenges of discrete and arbitrary medical knowledge such as high dimensionality, temporality, sparsity, irregularity and bias
          
          \item Explored a framework using the gradient of the attacked classifier to guide the greedy search for generating discrete white-box adversarial attacks on the medical predictive classifier 
          
          \item Conducted a comprehensive literature survey on the basic approaches to adversarial attacks generation and adversarial machine learning
          
        \end{itemize}
        
    \resumeSubheading
      {\href{https://isg.ics.uci.edu/}{Information Systems Group (ISG)}, University of California, Irvine}{Irvine, California}
      {Research Assistant, Advisor: \href{https://chenli.ics.uci.edu/}{Professor Chen Li}}{Jul. 2019 - Sept. 2019}
      {\textit{Wildfire Detection using Social Media}, [\href{https://github.com/ISG-ICS/Wildfires}{Github}], [\href{http://wildfires.ics.uci.edu:2333/}{Website}]}{}
      
        \begin{itemize}
          \item Collected and preprocessed large scale real-time tweets with corresponding text and images from twitter and regularly updated environmental information
          \item Completed twitter image classifier through convolutional neural networks (CNN) to detect highly wildfire-related images from real-time tweets out of efficiency and effectiveness
          \item Accomplished a CNN text classification architecture based on pre-trained word embedding to screen out expected tweets indicating real wildfire events
          \item Designed a fully convolutional network (FCN) framework to generate wildfire risk map considering several independent environmental infomation
          \item Coauthored two abstracts ``Using Social Media and Environmental Data for Wildfire Detection Based on Machine Learning Techniques'' and ``Identifying Wildfire Tweets by Semantic Analysis and Integrating Environmental Data Using Machine Learning'', published on \href{https://www.agu.org/fall-meeting}{AGU Fall Meeting 2019}, San Francisco, California
        \end{itemize}
        
    \resumeSubheading
      {DBGroup, SUSTech}{Shenzhen, China}
      {Research Assistant, Advisor: \href{https://acm.sustech.edu.cn/btang/}{Professor Bo Tang}}{Sept. 2018 - Feb. 2019}
      {\textit{News Headlines Generation with a Data-driven Approach}}{}
      
        \begin{itemize}
          \item Conducted training dataset of NBA games via collecting news headlines and corresponding data information in tuple format
          \item Explored top-k newsworthy fact from multidimensional dataset through algorithms such as skyline algorithm and longest streak discovery algorithm
          \item Compared effectiveness of different techniques such as Skip-Gram, CBOW and GloVe to train the word embedding for news headlines corpus 
          \item Implemented a sequence-to-sequence architecture using LSTM and GRU for generation from tuple data to descriptive text as the news headlines, combined with techniques such as attention mechanism and copy mechanism
        \end{itemize}
  \resumeSubHeadingListEnd
  
%-----------PROJECTS-----------------
\section{Selected Projects}
  \resumeSubHeadingListStart
  
  \resumeSubItem
      {Hello SUSTech [\href{https://github.com/RainyTong/Hello-SUSTech}{Github}]}{}
       \begin{itemize}
       \item Developed an Android app applied deep learning techniques for SUSTecher with the innovation point of positioning campus buildings by images and navigating user to the building
       \item Established and labeld training dataset for campus buildings classifier manually through extracting frames from video
       \item Achieved reletive high accuracy of above 94 percent to detect exact campus building given an uploaded image by designing a CNN classifier modified from widely used VGG framework
       \end{itemize}
       
   \resumeSubItem
      {SUSTech Lambda [\href{https://github.com/Henrycobaltech/SUSTechLambda}{Github}]}{}
       \begin{itemize}
       \item Developed an easily customized web platform to upload, execute and share scripts(e.g., Python, Bash, JavaScript) online which supports authority and group management
       \item Deployed web development techniques such as React, MongoDB, JAVA Spring Boot and Kotlin
       \end{itemize}
       
   \resumeSubItem
       {PintOS Threads \& User Programs \& File System}{}
       \begin{itemize}
       
       \item Enabled programs to interact with the OS via system calls by implementing Argument Passing, Process Control Syscalls, and File Operation Syscalls on a half-developed kernel
       \item Developed a Round-Robin scheduler and Priority scheduler based on time slice
       \item Deployed Bochs and QEMU simulators running on Linux
       \end{itemize}
  
  
%     \resumeSubItem{Hello SUSTech [\href{https://github.com/RainyTong/Hello-SUSTech}{Github}]} 
%       {An Android app for people in the campus of SUSTech including multiple fuctions, including an innovatisearch usage of searching school buildings by picture and navigating user to that building.}
%     \resumeSubItem{SUSTech Lambda [\href{https://github.com/Henrycobaltech/SUSTechLambda}{Github}]}
%       {A web platform to run and share scripts, which is easily customized and supports authority and group management. Techniques include React, MongoDB, JAVA Spring Boot, Kotlin and so on.}
%     \resumeSubItem{PintOS Threads \& User Program}
%       {Advanced functions implemented on a half developed operating system kernel, such as alarm clock, priority scheduling, argument passing and system calls.}
%      \resumeSubItem{Live Streaming Website Using DASH [\href{https://github.com/RainyTong/Live-Streaming-Website-using-DASH}{Github}]}
%       {A live streaming website designed to switch to different quality version of video according to the Internet speed. In this project, DASH is used for browser and RTMP is used for streamer.}
%       \resumeSubItem{In uence Maximization Problem(IMP)}
%         {A solution for NP-hard problem IMP in near-linear time using IMM algorithm based on martingales, a classic statistical tool. (Inspired by "In uence Maximization in Near-Linear Time: A Martingale Approach", Tang et al., SIGMOD’15.)}
%       \resumeSubItem{Capacitied Arc Routing Problem (CARP)}
%         {A solution for CARP based on path scanning algorithm with ellipse rule for initialization and tabu search algorithm for optimization. (Inspired by "A deterministic tabu search algorithm for the capacitated arc routing problem", Brandão et al., 2006.)}
        
%       \resumeSubItem{Playing Tetris on ARM Cortex-M3 Processor [\href{https://github.com/DennielZhang/SUSTech-CS301-Embeded-System-Project}{Github}]}
%       {A game of Tetris designed and built on ARM Cortex-M3 processor which implements basic game rules and can record game rankings of di erent users.}
      
        
  \resumeSubHeadingListEnd
%
%--------PROGRAMMING SKILLS------------
\section{Programming Skills}
 \resumeSubHeadingListStart
   \item{
      \textbf{Languages}{: Python, Java, C/C++, SQL, Assembly, LaTex, Javascripts}
   }
   \item{
      \textbf{Technologies}{: Git, PyTorch, Tensorflow, React, Angular}
   }
   \item{
      \textbf{Platforms}{: Linux, OSX, Windows}
   }
 \resumeSubHeadingListEnd
 
%--------HONORS & AWARDS--------------------
% \section{Honors \& Awards}
% \resumeSubHeadingListStart
%    \item{
%       \textbf{Second Class Scholarship}{: 2017-2018}
%    }
%    \item{
%       \textbf{MCM/ICM, Honorable Mention}{: Jan. 2018}
%    }
% \resumeSubHeadingListEnd
   


%-------------------------------------------
\end{document}
