%-------------------------
% Resume in Latex
% Author : Yutong Wang
% License : MIT
%------------------------
\documentclass[letterpaper,11pt]{article}
\usepackage{latexsym}
\usepackage[empty]{fullpage}
\usepackage{titlesec}
\usepackage{marvosym}
\usepackage[usenames,dvipsnames]{color}
\usepackage{verbatim}
\usepackage{enumitem}
\usepackage[hidelinks]{hyperref}
\usepackage{fancyhdr}
\usepackage[english]{babel}
\pagestyle{fancy}
\fancyhf{} % clear all header and footer fields
\fancyfoot{}
\renewcommand{\headrulewidth}{0pt}
\renewcommand{\footrulewidth}{0pt}
% Adjust margins
\addtolength{\oddsidemargin}{-0.5in}
\addtolength{\evensidemargin}{-0.5in}
\addtolength{\textwidth}{1in}
\addtolength{\topmargin}{-.5in}
\addtolength{\textheight}{1.0in}
\urlstyle{same}
\raggedbottom
\raggedright
\setlength{\tabcolsep}{0in}
% Sections formatting
\titleformat{\section}{
  \vspace{-4pt}\scshape\raggedright\large
}{}{0em}{}[\color{black}\titlerule \vspace{-5pt}]
% Hyperlink color
\usepackage{hyperref}
\hypersetup{
    colorlinks=true,
    linkcolor=blue,
    filecolor=blue,      
    urlcolor=blue,
    citecolor=blue,
}
%-------------------------
% Custom commands
\newcommand{\resumeItem}[2]{
  \item\small{
    \textbf{#1}{: #2 \vspace{-2pt}}
  }
}
\newcommand{\resumeSubheading}[6]{
  \vspace{-1pt}\item
    \begin{tabular*}{0.97\textwidth}[t]{l@{\extracolsep{\fill}}r}
      \textbf{#1} & #2 \\
%       \textbf{\small#3} & \textbf{\small #4} \\
      #3 & #4 \\
      \textbf{#5} & \textbf{#6} \\
    \end{tabular*}\vspace{-5pt}
}

\newcommand{\resumeSubItem}[2]{\resumeItem{#1}{#2}\vspace{-4pt}}
\renewcommand{\labelitemii}{$\circ$}
\newcommand{\resumeSubHeadingListStart}{\begin{itemize}[leftmargin=*]}
\newcommand{\resumeSubHeadingListEnd}{\end{itemize}}
\newcommand{\resumeItemListStart}{\begin{itemize}}
\newcommand{\resumeItemListEnd}{\end{itemize}\vspace{-5pt}}
%-------------------------------------------
%%%%%%  CV STARTS HERE  %%%%%%%%%%%%%%%%%%%%%%%%%%%%
\begin{document}
%----------HEADING-----------------
\begin{tabular*}{\textwidth}{l@{\extracolsep{\fill}}r}
  \textbf{\Large Yutong Wang} & Email : ytwwang@ucdavis.edu\\
  \href{https://rainytong.github.io/}{https://rainytong.github.io/} & Mobile : +86 15610502699 \\
\end{tabular*}
%-----------EDUCATION-----------------
\section{Education}
  \resumeSubHeadingListStart
    \resumeSubheading
        {University of California, Davis}{Davis, California}
        {Incoming Ph.D. in Computer Science}{Sept. 2020 -- Jun. 2025 (Expected)}
        {}{}
        
    \resumeSubheading
      {Southern University of Science and Technology (\href{https://www.sustech.edu.cn/en/}{SUSTech})}{Shenzhen, China}
      {B.E. in Computer Science and Engineering}{Sept. 2016 -- Jun. 2020 (Expected)}
      {}{}
      \begin{itemize}
      \item GPA: 89/100 (total); 92/100 (sophomore \& junior)
%       \\ 
%       GPA : 
      \item Annual Outstanding Student Scholarship (Top 8\%): 2017-2019
      
       \end{itemize}
       
    \resumeSubheading
      {University of California, Irvine}{Irvine, California}
      {Visiting Student in Information and Computer Sciences}{Jul. 2019 -- Sept. 2019}
      {}{}
       \begin{itemize}
        \item GPA (research evaluation): 4.0/4.0
       \end{itemize}
  \resumeSubHeadingListEnd

%-----------RESEARCH INTERESTS---------
% \section{Research Interests}
% Adversarial Machine Learning, Cybersecurity


 

%--------EXPERIENCE------------
\section{Experience}
  \resumeSubHeadingListStart
    \resumeSubheading
    {Teaching Assistant, SUSTech}{Feb. 2020 - Jun. 2020}
      {Assisted in teaching the course Deep Learning}{}
      {}{}
      
    \resumeSubheading
    {Research Intern, King Abdullah University of Science and Technology}{Sept. 2019 - Jan. 2020}
      {Led a research project focusing on the adversarial machine learning}{}
      {}{}
     
     \resumeSubheading
    {Research Assistant, University of California Irvine}{Jul. 2019 - Sept. 2019}
      {Worked on the California Wildfire project}{}
      {}{}
      
     \resumeSubheading
    {Research Assistant, SUSTech}{Jul. 2018 - Jun. 2019}
      {Led a research project focusing on news headlines generation in NLP}{}
      {}{}

  \resumeSubHeadingListEnd


%--------Publications------------
\section{Publications}
 \resumeSubHeadingListStart

   \item{First-author paper ``Attackability Characterization of Adversarial Evasion Attack on Discrete Data'', in submission
   
   \item{Coauthored abstract ``Using Social Media and Environmental Data for Wildfire Detection Based on Machine Learning Techniques'', accepted by \href{https://www.agu.org/fall-meeting}{AGU Fall Meeting 2019}, San Francisco, California}
   
   \item{Coauthored abstract ``Identifying Wildfire Tweets by Semantic Analysis and Integrating Environmental Data Using Machine Learning'', accepted by \href{https://www.agu.org/fall-meeting}{AGU Fall Meeting 2019}, San Francisco, California}
       

\resumeSubHeadingListEnd



%-----------Research EXPERIENCE-----------------
\section{Research Experience}
  \resumeSubHeadingListStart
    \resumeSubheading
      {\href{https://mine.kaust.edu.sa/Pages/Home.aspx}{MINE Lab}, King Abdullah University of Science and Technology}{Thuwal, Saudi Arabia}
      {Research Intern, Advisor: \href{https://mine.kaust.edu.sa/Pages/ZhangX.aspx}{Professor Xiangliang Zhang}}{Sept. 2019 - Jan. 2020}
      {\textit{Discrete Adversarial Attacks on Binary Feature Classifier}}{}
      
        \begin{itemize}
%           \item Achieved a higher performance on the state-of-the-art LSTM medical predictive model via designing a novel embedding with attention mechanism
            \item Achieved state-of-the-art prediction performance on medical dataset using a binary feature classification model based on Long Short-Term Memory (LSTM)
          
%           \item Leveraged heuristic clustering algorithms to analyze the quality of embedding learned for discrete knowledge

          \item Conducted a comprehensive literature review on the baseline approaches to adversarial attacks generation and adversarial machine learning
          
          \item Explored an innovative framework using the gradient of the attacked classifier to guide the greedy search for discrete attack generation
          
        \end{itemize}
        
    \resumeSubheading
      {\href{https://isg.ics.uci.edu/}{Information Systems Group (ISG)}, University of California, Irvine}{Irvine, California}
      {Research Intern, Advisor: \href{https://chenli.ics.uci.edu/}{Professor Chen Li}}{Jul. 2019 - Sept. 2019}
      {\textit{Wildfire Detection using Social Media}, [\href{https://github.com/ISG-ICS/Wildfires}{Github}], [\href{http://wildfires.ics.uci.edu:2333/}{Website}]}{}
      
        \begin{itemize}
          \item Collected and preprocessed regularly updated environmental information and large-scale real-time tweets with text and images from Twitter
          
          \item Implemented Text CNN tweets classifier, VGG Net for image classification and conducted comprehensive experiments on self-built Twitter dataset which achieved wildfire detection through social media
          
          \item Designed a Fully Convolutional Network (FCN) framework to generate wildfire risk map using environmental information
          
          \item Explored sentiment analysis for tweets via Stanford CoreNLP and AllenNLP frameworks
        \end{itemize}
        
    \resumeSubheading
      {DBGroup, SUSTech}{Shenzhen, China}
      {Research Assistant, Advisor: \href{https://acm.sustech.edu.cn/btang/}{Professor Bo Tang}}{Sept. 2018 - Feb. 2019}
      {\textit{News Discovery from Database and Headlines Generation}}{}
      
        \begin{itemize}
          \item Constructed a dataset of NBA news headlines and collected corresponding data information
          \item Explored top-k newsworthy fact from multidimensional dataset by implementing and comparing algorithms such as the skyline algorithm and longest streak discovery algorithm
          \item Compared effectiveness of different natural language processing techniques such as Skip-Gram, CBOW and GloVe to train the word embedding for news headlines corpus
          \item Implemented an LSTM-based table-to-sequence model for news headline generation leveraging attention mechanism and copy mechanism
        \end{itemize}
  \resumeSubHeadingListEnd
  
 

%--------PROGRAMMING SKILLS------------
\section{Skills}
 \resumeSubHeadingListStart
%    \item{
%       \textbf{Selected Core Courses}{: Software Engineering, Operating Systems, Computer Networks, Algorithm Design and Analysis, Artificial Intelligence, Deep Learning}
%    }
   \item{
      \textbf{Development}{: Java, Python, C/C++, SQL, Javascripts, HTML}
   }
   \item{
      \textbf{Deep Learning Frameworks}{: PyTorch, Tensorflow}
   }
   
 \resumeSubHeadingListEnd
 
 
  
%-----------PROJECTS-----------------
\section{Notable Course Projects}
  \resumeSubHeadingListStart
  
  \resumeSubItem
       {PintOS Threads \& User Programs \& File System}{}
       \begin{itemize}
       
       \item Improved a half-developed system kernel by implementing Parameters Passing and File Management system calls %Process Control system calls, and File Management system calls
%        \item Developed Round-Robin scheduler and Priority scheduler based on time slice
%        \item Deployed an advanced Linux filesystem
       \end{itemize}
  
  \resumeSubItem
      {Campus Android App, HelloSUSTech [\href{https://github.com/RainyTong/Hello-SUSTech}{Github}]}{}
       \begin{itemize}
%        \item Designed a Deep Learning Android App supporting functions like map localization, navigation, building recognition,
% class management
       \item Designed an Android App supporting functions like navigation, building recognition %map localization, navigation, building recognition, class management by utilizing Deep Learning models 
%        \item Involved techniques: Volley Web Framework, Gaode(AMap) Maps Framework, Spring Boot, PyTorch
       \end{itemize}
       
   \resumeSubItem
      {A Web Platform for Running Scripts, SUSTechLambda [\href{https://github.com/Henrycobaltech/SUSTechLambda}{Github}]}{}
       \begin{itemize}
%        \item Designed a web platform supporting uploading, running, sharing scripts for script language like: Python,
% Bash, JavaScript
       \item Designed a web platform supporting uploading, running, sharing scripts for script languages
%        \item Involved techniques: React.js, Spring Boot, Docker
       \end{itemize}
       
   
       
%    \resumeSubItem
%       {}{}
%       \begin{itemize}
      
%       \item 
      
%       \end{itemize}
      
  
  
%     \resumeSubItem{Hello SUSTech [\href{https://github.com/RainyTong/Hello-SUSTech}{Github}]} 
%       {An Android app for people in the campus of SUSTech including multiple fuctions, including an innovatisearch usage of searching school buildings by picture and navigating user to that building.}
%     \resumeSubItem{SUSTech Lambda [\href{https://github.com/Henrycobaltech/SUSTechLambda}{Github}]}
%       {A web platform to run and share scripts, which is easily customized and supports authority and group management. Techniques include React, MongoDB, JAVA Spring Boot, Kotlin and so on.}
%     \resumeSubItem{PintOS Threads \& User Program}
%       {Advanced functions implemented on a half developed operating system kernel, such as alarm clock, priority scheduling, argument passing and system calls.}
%      \resumeSubItem{Live Streaming Website Using DASH [\href{https://github.com/RainyTong/Live-Streaming-Website-using-DASH}{Github}]}
%       {A live streaming website designed to switch to different quality version of video according to the Internet speed. In this project, DASH is used for browser and RTMP is used for streamer.}
%       \resumeSubItem{In uence Maximization Problem(IMP)}
%         {A solution for NP-hard problem IMP in near-linear time using IMM algorithm based on martingales, a classic statistical tool. (Inspired by "In uence Maximization in Near-Linear Time: A Martingale Approach", Tang et al., SIGMOD’15.)}
%       \resumeSubItem{Capacitied Arc Routing Problem (CARP)}
%         {A solution for CARP based on path scanning algorithm with ellipse rule for initialization and tabu search algorithm for optimization. (Inspired by "A deterministic tabu search algorithm for the capacitated arc routing problem", Brandão et al., 2006.)}
        
%       \resumeSubItem{Playing Tetris on ARM Cortex-M3 Processor [\href{https://github.com/DennielZhang/SUSTech-CS301-Embeded-System-Project}{Github}]}
%       {A game of Tetris designed and built on ARM Cortex-M3 processor which implements basic game rules and can record game rankings of di erent users.}
      
        
  \resumeSubHeadingListEnd
%   \resumeItemListEnd



%--------HONORS & AWARDS--------------------
% \section{Honors \& Awards}
% \resumeSubHeadingListStart
%    \item{
%       \textbf{Second Class Scholarship}{: 2017-2018}
%    }
%    \item{
%       \textbf{MCM/ICM, Honorable Mention}{: Jan. 2018}
%    }
% \resumeSubHeadingListEnd
   


%-------------------------------------------
\end{document}
